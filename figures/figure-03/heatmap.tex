\begin{tikzpicture}
  \begin{axis}[
    %colorbar horizontal,
    colormap/jet,
    colormap name=mycolormap,
    colorbar,
    colorbar style={%
      title={$\log_{10}$(Flux)},
      title style={yshift=-0.5ex},
      yticklabel style={font=\small},
      yticklabel=$10^{\pgfmathprintnumber{\tick}}$,
      yticklabels={, -6, -4, -2, NaN}, % check based on above line
    },
      width=33cm,
      height=2.4cm,
      xlabel style={font=\Large},
      xtick={},
      xtick style={draw=none},
      ytick={1, 2},
      ytick style={draw=none},
      yticklabels={Wildtype, Disease},
      enlargelimits=false,
      axis on top,
      point meta min=-6,
      point meta max=-0.0001,
      mesh/cols=69,
      mesh/rows=2,
      %xmin=0,
      %xmax=69,
      xtick={%
        1, 2, 3, 4, 5, 6, 7, 8, 9, 10,
        11, 12, 13, 14, 15, 16, 17, 18, 19, 20,
        21, 22, 23, 24, 25, 26, 27, 28, 29, 30,
        31, 32, 33, 34, 35, 36, 37, 38, 39, 40,
        41, 42, 43, 44, 45, 46, 47, 48, 49, 50,
        51, 52, 53, 54, 55, 56, 57, 58, 59, 60,
        61, 62, 63, 64, 65, 66, 67, 68, 69
      },
      xticklabels={%
        2, 3, 4, 5, 6, 8, 9, 10, 11, 12, 17, 18, 19, 20, 25, 26, 27, 28, 29, 30,
        31, 32, 33, 46, 48, 62, 63, 64, 65, 66, 67, 68, 69, 41, 42, 43, 44, 52,
        53, 54,
        55, 57, 58, 59, 60, 47, 49, 50, 1, 7, 13, 14, 15, 16, 21, 22, 23, 24,
        34, 35,
        36, 37, 38, 39, 40, 45, 51, 56, 61
      },
      x tick label style={rotate=0, font=\footnotesize},
      xlabel={Reaction \#},
      xlabel style={yshift=-2.0ex},
  ]
  \addplot[matrix plot*,point meta=explicit]
      table [x=x,y=y,meta=z, col sep=comma] {/home/jchitpin/Documents/PhD/Projects/reproduce-efm-paper-2023/data/fluxes-heatmap-log10-scale.csv};
  \end{axis}
  % X-axis sub labels
  \node[] (ER1) at (rel axis cs: 0.0,-0.4) {};
  \node[] (ER2) at (rel axis cs: 4.4,-0.4) {};
  \node[font=\small,yshift=-0.0cm,align=center,anchor=north] (ER3) at ($(ER1)!0.5!(ER2)$) {Endoplasmic\\reticulum};
  \draw[-] (ER1.center) -- (ER2.center);
  \node[] (N1) at (rel axis cs: 4.6,-0.4) {};
  \node[] (N2) at (rel axis cs: 9.0,-0.4) {};
  \node[font=\small,yshift=-0.0cm,anchor=north] (N3) at ($(N1)!0.5!(N2)$) {Nucleus};
  \draw[-] (N1.center) -- (N2.center);
  \node[] (M1) at (rel axis cs: 9.2,-0.4) {};
  \node[] (M2) at (rel axis cs: 12.7,-0.4) {};
  \node[font=\small,yshift=-0.0cm,anchor=north] (M3) at ($(M1)!0.5!(M2)$) {Mitochondria};
  \draw[-] (M1.center) -- (M2.center);
  \node[] (OIM1) at (rel axis cs: 12.9,-0.4) {};
  \node[] (OIM2) at (rel axis cs: 29.1,-0.4) {};
  \node[font=\small,yshift=-0.0cm,anchor=north] (OIM3) at ($(OIM1)!0.5!(OIM2)$) {Outer \& inner membrane};
  \draw[-] (OIM1.center) -- (OIM2.center);
  \node[] (GA1) at (rel axis cs: 29.3,-0.4) {};
  \node[] (GA2) at (rel axis cs: 40.0,-0.4) {};
  \node[font=\small,yshift=-0.0cm,anchor=north] (GA3) at ($(GA1)!0.5!(GA2)$) {Golgi apparatus \& cytoplasmic face};
  \draw[-] (GA1.center) -- (GA2.center);
  \node[] (L1) at (rel axis cs: 40.2,-0.4) {};
  \node[] (L2) at (rel axis cs: 42.7,-0.4) {};
  \node[font=\small,yshift=-0.0cm,anchor=north] (L3) at ($(L1)!0.5!(L2)$) {Lysosome};
  \draw[-] (L1.center) -- (L2.center);
  \node[] (C1) at (rel axis cs: 42.9,-0.4) {};
  \node[] (C2) at (rel axis cs: 62.8,-0.4) {};
  \node[font=\small,yshift=-0.0cm,anchor=north] (C3) at ($(C1)!0.5!(C2)$) {Cell};
  \draw[-] (C1.center) -- (C2.center);

\end{tikzpicture}

